\chapter{Anforderungen}
Im folgenden sind die Anforderungen an die Software spezifiziert. Soweit möglich wurden
die Anforderungen schon in einzelne Komponenten und Module gegliedert. Eine grobe Übersicht gibt auch Diagramm \ref{spec:dia:moduls}.

Alle Anforderungen werden mit einer Kennnummer in der Form 
\[ /<Gruppennummer> . <Anforderungsnummer> [. <Unternummer> ]/ \]
gekennzeichnet, damit diese später wieder referenziert werden können.

\section{Crawlermodul}
\initReqgrp{crawler}
Dieses Modul soll als eigenständiger Prozess laufen und in regelmäßigen Abständen Crawlvorgänge starten
und die Daten in das Archiv schreiben.

\subsection{Steuerung}
\begin{description}
	\item [\req{Config-file}{conf}]
		Die Steuerung des bzw. der Crawler erfolgt über eine Config-Datei.
		Darin werden bereits Default- und Fallbackwerte festgelegt  
		Es können folgende Parameter eingestellt werden:
		\begin{description}
			\item [\subreq{Suchtiefe}{depth}]
				Suchtiefe bis zu der Links gefolgt werden soll.
			\item [\subreq{Suchintervall}{tinterval}]
				Zeitintervalle der Crawlvorgänge.
			\item [\subreq{Maximale Instanzen}{maxInst}]
				maximale Anzahl der gleichzeitig gestarteten Crawlerinstanzen.
			\item [\subreq{Filtereinstellungen}{filter}]
				Liste von Verwendeten Ausschlussfiltern und deren Modulpfad.
		\end{description}
	\item [\req{Kommandozeileninterface}{cmd}]
		Optional können beim Start mittels Kommandozeile zusätzliche Parameter übergeben werden.
		\begin{description}
			\item [\subreq{Überschreiben}{overwrite}]
				Werte aus dem Config-file werden damit überschrieben.
			\item [\subreq{Domainliste}{list}]
				Es kann eine Liste von Domains übergeben werden, 
				die als Startpunkte für die Crawler verwendet werden sollen.
				Die Übergabe von mindestens einem Element ist aber immer notwendig.
			\item [\subreq{DB-Recovery}{recover}]
				Es kann ein Datenbank-Recovery erzwungen werden. Siehe auch \reqref{req:db:recovery}
		\end{description}
\end{description}

\subsection{Ausführung des Crawlvorgangs}
\begin{description}
	\item [\req{Parallelität}{concurrent}]
		Die Ausführung der Crawlvorgänge soll parallel durchgeführt werden.
	\item [\req{Instanziierung}{instance}] 		
		Pro angegebener URL wird eine Crawlerinstanz gestartet bis die Obergrenze an Instanzen  
		(siehe \reqref{req:crawler:conf:maxInst}) erreicht wird.
		Um Überschneidungen zu vermeiden, werden Domainnamen anderer Instanzen ausgeschlossen.
		Für die Crawlerinstanzen wird ein externes Tool verwendet (wget).
	\item [\req{Crawlen}{crawl}]
		Jede gestartete Instanz beginnt nun den Crawlvorgang.
		\begin{description}
			\item [\subreq{Herunterladen}{download}]
				Jede Instanz kopiert die heruntergeladene HTML-Dateien 
				der Seite in ein temporäres Verzeichnis je Crawlerinstanz.
			\item [\subreq{Ordnerstruktur}{structure}]
				Dabei wird die online vorhandene URL-Pfadstruktur der HTML-Dateien 
				auf das Dateisystem abgebildet.
				Je Domain wird dadurch ein Hauptverzeichnis im TMP-Ordner erzeugt.
		\end{description}
	\item [\req{Filterung}{filter}]
		Für alle HTML-Dateien wird eine Liste von Filter durchlaufen.
		Jeder Filter prüft, ob die HTML-Datei behalten oder verworfen werden soll.
		Verworfene Dateien werden sogleich gelöscht.
	\item [\req{Bereinigung}{clean}]
		In diesem Teil werden die heruntergeladenen Ordner bereinigt,
		also leere Ordner oder Nicht-HTML-Dateien gelöscht.
	\item [\req{Normalisierung}{normalize}]
		Es wird ein Archivordner (im Folgenden \htmlarc genannt) 
		mit dem Namen des HTML-files (inklusive Dateiendung) 
		erzeugt, das HTML-file in ''data.html'' umbenannt und 
		in das soeben erzeugte HTML-Archiv verschoben.
	\item [\req{Extraktion der Metadaten}{metaextract}]
		In diesem Vorgang werden die im Datenmodell \ref{spec:model} definierten Metadaten 
		aus dem HTML extrahiert und zwischengespeichert. Es wird auch der
		CommitTag erzeugt, wobei der Zeitpunkt für alle Dateien pro Domain und Crawler-Instanz gleich ist.
	\item [\req{Erzeugung von XML-Dateien}{xml}]
		Die zwischengespeicherten Metadaten werden in XML-Metadatein (siehe auch \ref{spec:xml}) 
		geschrieben und jeweils im zugehörige \htmlarc gespeichert. 
	\item [\req{Synchronisation}{sync}]
		Die vorbereiteten Daten in den TMP-Ordnern werden nun in das vorhandene Archiv synchronisiert 
		(mit rsync). 
		Veraltete Domain-Ordner werden dabei komplett überschrieben, 
		können aber gegebenfalls mithilfe der Versionierung wiederhergestellt werden. 
		Dabei sollen auch veraltete Analysedaten gelöscht werden.
		(Siehe auch \ref{spec:filearchive})
	\item [\req{Datenbankaktualisierung}{dbupdate}]
		Die Datenbank wird nun mithilfe der zwischengespeicherten Metadaten aktualisiert.
\end{description}

\section{Filtermodule}
\initReqgrp{filter}
\begin{description}
	\item [\req{Schnittstelle}{interface}]
		Alle Filtermodule sollen eine vorgegebene Schnittstelle erfüllen.
	\item [\req{Konfiguration}{conf}]
		Alle Module müssen wie unter \reqref{req:crawler:conf:filter} beschrieben dem Crawlmodul bekanntgemacht werden.
		Eine Zentrale Speicherung der einzelnen Module in einem Filterverzeichnis ist anzustreben.
	\item [\req{HTML Überprüfung}{htmlchk}]
		Ein Filter soll ein gegebenes HTML-File überprüfen und einen Wahrheitswert zurückgeben,
		ob dieses behalten oder verworfen werden soll.
		Die Prüfmethode ist dabei vom einzelnen Zweck des Filters abhängig.
	\item [\req{Implementierung Testfilter: Werbefilter}{testfilter}]
		Als Testfilter wird ein Filter implementiert, 
		der mittels einer Blacklist HTML-Dateien bestimmter Werbedomains aussortiert.
\end{description}

\section{Dateiarchiv} \label{spec:filearchive}
\initReqgrp{arc}
\begin{description}
	\item [\req{Struktur}{structure}]
		Die Dateistruktur wird bereits vom Crawlvorgang vorgegeben, wird hier aber nochmal erläutert:
		\begin{itemize}
			\item Auf der obersten ebene stehen die Domain-Ordner.
			\item Darunter wird die von der URL abgebildeten Struktur
				innerhalb der Domain nachgebildet.
			\item Die einzelnen HTML-Datein werden durch \htmlarc-Ordner ersetzt.
			\item Jedes \htmlarc enthält die Quell-HTML-datei, 
				die aber in data.html unbenannt wurde
				sowie die dazugehörige XML-Datei (data.xml). 
				Ein \htmlarc kann aber auch weitere Dateien enthalten, welche nach dem
				Crawlen hinzugefügt werden können.
		\end{itemize}
	\item [\req{Berechtigungen}{rights}]
		Generell dürfen keine Änderungen an bestehenden Dateien durchgeführt werden 
		(Ausname: siehe \ref{spec:xml}).
		\begin{description}
			\item [\subreq{Externe Benutzer}{extern}]
				Von exterenen Nutzern (Java-Clients) dürfen nur Dateien in \emph{aktuelle} \htmlarc-Ordner
				hinzugefügt werden.
			\item [\subreq{Crawler}{crawler}]
				Crawler dürfen neue \htmlarc-Ordner hinzufügen oder alte löschen bzw. ersetzen, 
				wobei immer das gesamte \htmlarc ersetzt wird. 
		\end{description}
	\item [\req{Synchronisation}{sync}]
		Um gleichzeitige Zugriffe auf Dateien zu verhindern, 
		müssen Teile des Archivs über Locks gesichtert werden.
		Dabei wird immer der gesamte Domain-Ordner gesperrt.
		Dabei muss vor jedem Lese- oder Schreibvorgang ein Lock gesetzt und
		beim Beenden des Vorgangs wieder entfernt werden.
	\item [\req{Versionierung}{version}]	
		Jeder Domainordner wird über eine dezentrale Versionsverwaltung (git) versioniert. 
		Damit ist das Wiederherstellen älterer Versionen grundsätzlich möglich,
		wobei diese aber manuell über die git-Schnittstellen abgerufen werden müssen.
		\begin{description}
			\item [\subreq{Hinzufügen}{add}]
				Beim Hinzufügen von neuen Dateien müssen diese der Versionsverwaltung bekanntgemacht werden.
			\item [\subreq{Commit}{commit}]
				Änderungen müssen stets mit einem Commit bestätigt werden.
				Dabei wird zur Identifikation immer ein bestimmter CommitTag verwendet.
				Dabei ist zu beachten:
				\begin{itemize}
					\item wurden Dateien während des Crawlvorgangs hinzugefügt,
						wird ein neuer CommitTag erzeugt.
					\item bei nachträglichen Änderung wird der alte CommitTag wiederverwendet.
				\end{itemize}
			\item [\subreq{Ändern und Löschen}{update}]
				Beim Überschreiben und Löschen von Dateien müssen keine besonderen Vorkehrungen
				getroffen werden.
				Diese werden von der Versionierung ,,in tieferen Ebenen'' gespeichert.
			\item [\subreq{Schreiben in veraltete Verzeichnisse}{writeToOldDirs}]
				Sollte in Ausnahmefällen in veraltete Verzeichnisse geschrieben werden, 
				zum Beispiel weil die Daten von den Crawlern während einer Analyse geändert wurden,
				dann werden die Daten verworfen und der Schreiber muss benachrichtigt werden.
				(z.B. mittels einer Exception)
		\end{description}
	\item [\req{Dateisystem}{fs}]
		Beim darunterliegenden Dateisystem wird von einem vorhandenen Unix-Filesystem ausgegangen.
	\item [\req{Komprimierung}{compress}]
		Eine explizite Dateikomprimierung wird nicht vorgesehen, 
		ist aber zum Teil schon durch die Versionierung gegeben, 
		da alte Revisionen gepackt abgelegt werden.
\end{description}


\section{Programmierschnittstelle - Java-Client}\label{spec:client}
\initReqgrp{client}
	Diese Schnittstelle soll die Anbindung der Analysemethoden ermöglichen und 
	macht gleichzeitig einen Zugriff über das Netzwerk möglich.
\begin{description}
	\item [\req{Konfiguration}{conf}]
		Es sind die IP und der Port des Servers in einer Config-Datei zu hinterlegen.
	\item [\req{Client-API}{api}]
		Für Benutzer des Clients wird eine Programmierschnittstelle in Java zur Verfügung gestellt.
		Die API umfasst dabei folgende Schnittstellen:
		\begin{description}
			\item [\subreq{MetaData}{meta}]
				Grundlegende Methoden einer Metadatenklasse.
				Neben den Metadateninformationen soll diese Klasse auch als
				Schlüsselelement zum Zugriff auf die Archivordner und XML-Dateien dienen.
			\item [\subreq{WebarchiveClient}{client}]
				Zentrale Schnittstelle zum Zugriff auf das Webarchiv, Details siehe unten.
			\item [\subreq{Observer}{observer}]
				Implementierungen dieses Observers können sich beim Client anmelden, 
				um über Änderungen informiert zu werden.
		\end{description}
	\item [\req{Registrierung am Server}{register}]
		Alle aktiven Java-Clients werden beim Server gespeichert, 
		um diese über Änderungen informieren zu können.
		Beim Abmelden oder Beenden muss ein Client aus dieser Registrierung gelöscht werden.
	\item [\req{Observer}{observer}]
		Benutzer des Clients können sich beim Client als Observer anmelden, 
		um über Änderungen informiert zu werden.
	\item [\req{Benachrichtigungen}{notifies}]
		Vom Server kommen in regelmäßigen Abständen Nachrichten über Änderungen.
		Diese werden an angemeldete Observer weitergegeben.
		Die Informationen bestehen dabei aus einer Liste von neuen CommitTags.
		Benutzer des Client können dann entscheiden, welche Daten Sie abrufen wollen.
	\item [\req{Datenbankabfragen}{dbquery}]
		Es sollen auch vorbereitete SQL-Statements an den Server geschickt werden können. 
		Die SQL-Abfrage wird soweit vorbereitet, 
		dass nur noch ein SQL-Bedingungsausdruck für die WHERE-Klausel angegeben werden muss.
		Optional soll auch eine ORDER-BY-Klausel im selben Stil angegeben werden können. 
		Als return-Wert wird eine Liste von Metadatenobjekten zurückgegeben.
	\item [\req{Datei Listing}{ls}]
		Mittels der Metadatenobjekte kann man sich über eine gesonderte Anfrage eine Auflistung
		über den Inhalt eines \htmlarc-Ordners zurückgeben lassen.
	\item [\req{Datei Lesen}{readFile}]
		Mittels Metadatenobjekt und Dateipfadangabe wird ein Dateistream zum Lesen zurückgegeben.
	\item [\req{Datei Schreiben}{writeFile}]
		Mittels Metadatenobjekt und Dateipfadangabe wird ein Dateistream zum Schreiben zurückgegeben.
		Wie oben beschrieben, dürfen dabei keine Dateien überschrieben werden. 
	\item [\req{Auslesen von zusätzlichen Tags}{selectTag}]
		Durch Übergabe eines Metadata-objekts und eines Tagnamens an eine get-Methode 
		wird ein passender XML-Node herausgesucht.
	\item [\req{Erweiterung	der XML-Daten}{addTag}]
		Mittels einer set-Methode, die Namen und Inhalt des Tags als Parameter erhält, 
		können neue Tags zur XML-Datei hinzugefügt werden.
	\item [\req{Test Analysetool}{testanalyzer}]
		Zur Demonstrations- und Testzwecken der Java-Clientschnittstelle wird ein Analysetool erstellt, 
		welches die Wörter im HTML zählt und das Ergebnis im Archiv 
		als Datei sowie im XML als zusätzliches Tag speichert.
	\end{description}

\section{Server} \label{spec:server}
\initReqgrp{server}
\begin{description}
	\item [\req{Konfiguration}{conf}]
		Es sind die IP und der Port des Servers in einer Config-Datei zu hinterlegen.
		Desweiteren werden darin auch alle Pfade zu externe Libraries, Treibern und Ressourcen
		hinterlegt.
	\item [\req{Client-Server Kommunikation}{comm}]
		Es muss ein Nachrichtensystem zwischen dem Client und dem Server implementiert werden.
		Über bestimmte Kennungen (z.B. über enums) ist der Inhalt einer Nachricht zu kennzeichnen.
		Dabei müssen folgende Informationen Ausgetauscht werden können:
		\begin{description}
	\item [\subreq{Austausch von Stream-Daten}{streams}]
		Zwischen C. und S. müssen Daten in Form von Streams verschickt werden können.
		Diese müssen in geeigneter Form bei der Übertragung gepuffert werden.
	\item [\subreq{Exceptions}{exceptions}]
		Vom Clientanwender verursachte Exceptions werden an diesen weitergeleitet und 
		im Client erneut geworfen.
	\item [\subreq{Datenbankabfragen}{dbquery}]
		Datenbankanfragen werden in Form von sql vom Client geschickt und diesem
		in Form von Metadaten-Objekten beantwortet.
	\item [\subreq{Änderungen im Archiv}{changes}]
		Ein Client wird über Änderungen im Archiv informiert.
\end{description}
	\item [\req{Clienten registrieren}{register}]
		Der Server hält eine Liste von angemeldeten Java-Clienten und 
		verwaltet die Verbindungen der Clienten und verwirft sie bei Verbindungsverlust oder 
		beim Beenden der Clients. 
	\item [\req{Update-Notifier}{notifier}]
		Der Update-Notifier ist ein eigens laufender Thread des Servers, der in einem vorgegebenen Intervall (z.B. stündlich) in der Datenbank prüft ob neue Commit-Tags vorhanden sind.
		Hierfür speichert er sich den Zeitpunkt der letzten Update-Suche und vergleicht ihn mit dem Datum der Datensätze in der Datenbank.
		Werden Änderungen gefunden, werden das Datum und der Commit-Tag zwischengepuffert.
		\begin{description}
			\item [\subreq{Konfiguration}{config}]
				Der Benachrichtigungsintervall muss in einem Configfile gespeichert werden.
			\item [\subreq{Clienten informieren}{notify}]
				Sobald die Update-Suche fertig ist, 
				werden die gepufferten Informationen an alle registrierten Clienten geschickt.
				Siehe auch \reqref{req:client:notify}
		\end{description}
\end{description}

\section{XML-Dateien} \label{spec:xml}
\initReqgrp{xml}
\begin{description}
	\item [\req{Inhalt}{content}]
		Die XML-Dateien enthalten in ihrem Wurzelknoten eine Meta- und einen Datenknoten.
		Der Metaknoten ist für die oben beschriebenen Metadaten reserviert.
		Der Datenknoten ist anfangs leer und kann von Benutzern um weitere Knöten erweitert werden.
	\item [\req{Validierung}{validate}]
		Für die Validierung der XML-Daten muss eine XML-Schema ausgearbeitet werden.
		Eine Validierung ist dann durchzuführen, nachdem eine XML-Datei erweitert worden ist.
		Dabei auftretende Fehler werden in eine Log-Datei geschrieben.
	\item [\req{Erweiterbarkeit}{extense}]
		Wie oben beschrieben ist beim Design auf Erweiterbarkeit zu achten.
		Der Inhalt des Knotens ist frei wählbar, der Name des Knotens muss aber
		eindeutig sein und darf mit vorhandenen Knöten nicht kollidieren
		(ggf. mit namespaces arbeiten).
		Sollten die XML-Dateien erweitert werden, so muss vom Benutzer auch das Schema erweitert werden.
	\item [\req{Schreibschutz}{writeprotect}]
		Bei schreibenden Zugriffen auf die XML-Datei ist darauf zu achten, 
		dass nur neue Knoten hinzugefügt werden dürfen und
		keine vorhandenen überschrieben oder geändert werden können.
		Desweiteren muss der Metaknoten geschützt werden.
		Es dürfen auch keine gleichnamigen Knöten in der obersten Ebene des Datenknotens
		vorhanden sein.
\end{description}
	
\section{Datenbank} \label{spec:db}
\initReqgrp{db}
\begin{description}
	\item [\req{Konfiguration}{conf}]
		Die Struktur der Datenbank wird als create-table statement in einer sql-Datei gespeichert.
		Falls nötig werden auch alle weiteren vorbereiteten sql-Dateien im selben Ordner abgelegt.
	\item [\req{Inhalt}{content}]
		Die Datenbank enthält nur eine Tabelle deren Zeilen die Metadaten 
		(siehe \ref{spec:model}) enthalten. 
		Es werden die Speicherstände aller HTML-Dateien und Versionen im Archiv festgehalten.
	\item [\req{Berechtigungen}{rights}]
		Schreibrechte werden nur dem Crawlmodul erteilt.
		Gelesen kann falls notwendig von allen Komponenten werden.
		Wobei externe Benutzer über die Schnittstellen der Clients select-statements abzusetzen können.
	\item [\req{Aktualisierung}{update}]
		Die Datenbank muss beim Fertigstellen des Crawlvorgangs auf den neuesten Stand gebracht werden.
	\item [\req{Wiederherstellung}{recovery}]
		Sollte die Datenbank beschädigt oder geändert werden, dann soll diese wieder aus den
		im Archiv vorhanden XML-Metadaten rekonstruiert werden können.
		Aus diesem Grund wird erstmal von einer einfachen SQLite-Datenbank ausgegangen, 
		da diese dateibasiert ist und daher relativ leicht erstellt und wieder gelöscht werden kann.
\end{description}
