\chapter{Datenmodell der Metadaten} \label{spec:model}
An dieser Stelle wird bereits ein vereinfachtes Datenmodell festgelegt, 
welches später auf Datenbank und XML-Daten umgesetzt wird und zentrale Metadaten 
der heruntergeladenen Dateien erfassen soll. 

Für jede Datei wird dabei ein solcher Satz von Metadaten angelegt.
Die Metadaten sollen vor allen Dingen Such- und Sortieraufgaben erleichtern. 

\begin{table}[h]
\centering
\begin{tabular}{|l|l|l|}	
	\hline
	Name 		& Datentyp 				& Beschreibung \\
	\hline
	url 		& String 				& Original URL der Datei\\
	\hline
	mimeType	& String 				& MIME-Typ der Datei\\
	\hline
	title 		& String 				& Soweit vorhanden, wird der Titel der Seite \\ 
	 			& 						& gespeichert \\ 
	\hline
	path 		& String 				& Dateipfad zum Archivordner der Datei im Webarchiv \\
	\hline
	domain 		& String 				& Name der Domain \\
	\hline
	crawlTime 	& Timestamp 			& Beginn des Crawls \\
	\hline
	createTime 	& Timestamp 			& Erzeugungszeitpunkt der Datei \\
	\hline
	commitTag 	& String 				& Der Committag dient zum Wiederauffinden \\
	 			& 		 				& in der Versionsverwaltung. \\ 
				& 						& Der Tag setzt sich aus domain und crawlTime zusammen.\\
				&  						& Form: \\
				&  						& commitTag := $<$crawlTime$>@<$domainName$>$ \\
	\hline
\end{tabular}
\caption{Metadaten}
\end{table}

\paragraph{Anmerkung Timestamp-typ:}
Alle Timestamps werden als String in der Form behandelt:
''yyyy-mm-dd hh:mm:ss'', bei der Speicherung im XML ist das Leerzeichen zwischen Datum und Zeit durch ein 'T' zu ersetzen. Als Zeitzone wird CET verwendet.

