\chapter{Datenmodell der Metadaten} \label{spec:model}
An dieser Stelle wird bereits ein vereinfachtes Datenmodell festgelegt, welches
später auf Datenbank und XML-Daten umgesetzt wird und zentrale Metadaten über HTML-Seiten erfassen soll. 

Für jede HTML-Datei wird dabei ein solcher Satz von Metadaten angelegt.
Die Metadaten sollen vor allen Dingen Such- und Sortieraufgaben erleichtern. 

\begin{table}[h]
\centering
\begin{tabular}{|l|l|l|}	
	\hline
	Name 		& Datentyp 				& Beschreibung \\
	\hline
	URL 		& String 				& Original URL der HTML-Datei\\
	Titel 		& String 				& Soweit vorhanden wird der Titel der Seite \\ 
	 			& 						& gespeichert \\ 
	Archivpfad 	& String 				& Dateipfad des \htmlarc-Ordners im Webarchiv \\
	Crawlzeit 	& Timestamp oder 		& Zeitpunkt des Crawls \\
				& Integer (UTC in ms) 	&  \\
	Commit Tag 	& String 				& Der Committag dient zum Wiederauffinden \\
	 			& 						& in der Versionsverwaltung. \\ 
				& 						& Der Tag setzt sich aus Domainnamme und \\
				& 						& dem Zeitpunkt des Crawlvorgangs zusammen.\\
	\hline
\end{tabular}
\caption{Metadaten}
\end{table}

