\chapter{Entwicklungsumgebung} \label{spec:req:devenv}
\section{Programmiersprachen}
	Es werden die Sprachen Python und Java benutzt.
	\subsection{Python}
		Python in der Version 2.7 oder 3 wird für die systemnahen Teile verwendet:
		\begin{itemize}
			\item Das gesamte Crawlermodul
			\item Die Filtermodule
			\item Der Zugriff auf das Dateisystem (Archiv) und die notwendige Ordnersynchronisation.
		\end{itemize}
		Python entält bereits leistungsfähige Libraries für den Zugriff auf das Dateisystem, SQLite und 
		die Versionsverwaltung git.
	\subsection{Java}
		Die Client-Server Architektur der oben genannten Programmierschnittstelle werden 
		mit Java 1.7 umgesetzt.
		Hierzu ist Java besonders geeignet und es ist gewährleistet, 
		eine an der Hochschule Hof allgemein verständliche Schnittstelle zu schaffen.
\section{Dokumentation}
	Die Dokumentation wird in \LaTeX als ein fortlaufendes Gesamtdokument erstellt, 
	welches je nach Phase um weitere Teile erweitert wird.
\section{Teamsynchronisation}
	Dokumente und Quellcode werden über ein gemeinsames Repository auf github.com synchronisiert.
\section{Sprache}	
	Die Sprache der Dokumentation ist Deutsch wobei natürlich geläufige Fremdwörter enthalten sind. 
	Quellcode, Kommentare und daraus abgeleitete APIs (Javadoc und Sphinx) sind in Englisch zu verfassen.
