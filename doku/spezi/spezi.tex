\chapter{Crawlermodul}

Die Steuerung erfolgt über ein Configfile (optional Kommandozeileninterface).
Es können folgende Parameter eingestellt werden:
	-Crawltiefe
	-Zeitintervalle der Crawlvorgänge
	-maximale Anzahl der gleichzeitig gestarteten Crawlerinstanzen
	-Domains, also die Startpunkte für die Crawler, pro Domain wird dann ein Crawler gestartet,
		bis die Obergrenze erreicht wurde.

Archiv
Das Verzeichnis ist in einzelne Domainordner getrennt. Jeder Domainordner wird über ein Versionsverwaltung
(z.B. git) versioniert. Damit ist das Wiederherstellen älterer Versionen grundsätzlich möglich.
Beim Dateisystem wird von einem Linuxsystem ausgegangen.

xml-Metadaten
Die xml-Datei enthält grundsätzliche Metainformationen über eine archivierte html-Datei.
Die xml-Datei soll auch nachträglich über eine Programmierschnittstelle um weitere xml-tags erweiterbar sein.
Folgende Informationen stehen bereits fest: 
- Git Commit Tag
- URL
- Dateipfad im Archiv
- Hashsumme über den Content

Datenbank
Die Datenbank dient zur Speicherung der grundlegenden Metadaten.
Diese soll während des Crawlvorgangs auf dem neuesten Stand gehalten werden.
Sollte die Datenbank beschädigt oder geändert werden, dann soll diese wieder aus den
xml-Metadaten rekonstruiert werden können.
 
Crawlvorgang
Für die Crawlerinstanzen wird ein externes Tool verwendet (z.B. wget).
Jede Instanz kopiert den Inhalt der Seite in je ein temporäres Verzeichnis.
Dabei sollen die html-Dateien je Domain in einen Ordner geschrieben werden.
Nach diesem Vorgang werden die tmp-Ordner bereinigt (z.B. leere Ordner entfernt)
und die html-Dateien in einen Ordner gleichen Namens kopiert.
Nun werden die Metadateien im xml-Format extrahiert und im o.g. Ordner gespeichert.
Zuletzt werden die so vorbereiteten tmp-Ordner in das vorhandene Archivverzeichnis
hineinsynchronisiert (z.B. mit rsync). Dabei wird jeder Domainordner über ein Dateimutex gelockt, um
gleichzeitiges Schreiben zu verhindern.
Zum Abschluss werden die Änderungen den Versionsverwaltungen der Domainordner mit einem Commit bestätigt.
Während oder nach dem Synchronisationsvorgang wird ein Datenbank-dump für die neuen oder geänderten Daten
erstellt. 

Programmierschnittstelle - Java-Client
Über diesen Client können vorbereitete Sql-Statements an einen Java-Server geschickt werden. Als return-Wert
wird eine Liste von Metadatenobjekten zurückgegeben.
Über diese Metadatenobjekte kann man sich nun über eine Anfrage die vorhandenen Archivordner aus dem Archiv herunterladen.
Auch eine Erweiterung der xml-Dateien um neue Tags bzw. die Abfrage von Tags soll über die 
Metadatenobjekte möglich sein.
Bei schreibendem Zugriff auf das Archiv muss immer auf das Setzen von Dateilocks geachtet werden.

% Diagramm

 
