\chapter{Programmstart und Benutzung}
	\section{Backend}
	\section{Frontend}
		\subsection{Start Server}
		Das Frontend (der Java-Server) kann über das mitgelieferte jar gestartet werden.
		Wird der Server über den source-code gestartet (bzw. compiliert) dann ist der Einstiegspunkt
		die Klasse \lstinline{webarchive.init.Launcher.java}.

		Beim Start muss der Pfad (i.d.R. absolut) des Config-files übergeben werden. 
		Dieses befindet sich normalerweise direkt im Installationsordner des Archivs.

		Damit der Server funktionsfähig ist muss mindestens der javadapter (siehe Backend)
		vorher gestartet werden.
		\subsection{Start Client}
		Ein Client kann mittels eines \lstinline{webarchive.api.WebarchiveClientFactory} erzeugt werden,
		wobei vorher in der Factory die Serveraddresse gesetzt muss.
		
		Über den Client können dann Abfragen gestartet oder Daten dem Archiv hinzugefügt werden. 
		Der Benutzer sollte sich vor der Benutzung mit der javadoc des api-Pakets und der Unterpakete vertraut machen,
		da hier alle Details zur Benutzung beschrieben sind.
		\paragraph{Anmerkung:} Werden über die Xml-Methoden Datenelemente hinzugefügt, 
		sollte auch umbedingt das sich im Archivordner befindliche \lstinline{file.xsd} (XML-Schema)
		angepasst werden.
	

