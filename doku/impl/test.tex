\chapter{Tests}
\section{ClientMockup}
	Der ClientMockup testet alle Schnittstellen zum Client, zum Backend (JavaDapter) sowie die Kommunkation übers Netzwerk.
	Dabei wird das Zusammenspiel aller Komponenten getestet.
\section{Unittests}
	In den Unittests wird die Funktionalität der einzelnen Klassen getestet.
	Die Tests sind packageweise in Testsuites verknüpft, welche wiederum über eine übergeordnete TestSuite aufgerufen werden können.
	Hier folgt die Erläuterung für jedes Paket:	
	\subsection{webarchive.api}
		In der API werden nur die konkreten Klassen getestet, Schnittstellen
		oder Abstraktionen werden in den konkreten Implementierungen der anderen Pakete oder im ClientMockup getestet.
		\subsubsection{.model}
		Bei den Modelklassen müssen hauptsächlich Getter getestet.
		Außerdem werden auch illegale Eingaben in die Konstruktoren getestet,
		wobei diese nur mit asserts abgefangen werden, da Objekte später nicht
		ins System geschrieben werden können.
		\subsubsection{.select}
		Die API-Selectklassen bereiten die Eingaben der Benutzer auf die
		Select-klassen im Server vor (benannte where-Parameter werden in
		eine generische Arrayform gebracht)
		Es wird deshalb getestet, ob durch Eingaben die richtige Arrayausgabeform erzeugt wird.  
		\subsubsection{.xml}
		Es wird die TagName-klasse getestet, welche dafür sorgt dass Xml-Tagnames mit richtigen Präfixen versehen wird. Es wird also die Erzeugung mit verschieden Eingaben getestet.
	\subsection{webarchive.xml}
		Da das XML-Paket hierarchisch aufgebaut ist sind die Tests dementsprechend gestaltet. 
In Lowlevel-Klassen werden die Primitivfunktionen getestet, während in den Highlevel-Controllern (XmlHandler, XmlEditor) zusammengesetzte Funktionalitäten getestet werden. 
	Die Netzwerkverbindung wird dabei mit Mockupklassen umgangen. 
	Es wird auch die Konfigurierung von XmlConf mit einer Test-Configdatei geprüft.

	\subsection{webarchive.dbaccess}
		a die  
