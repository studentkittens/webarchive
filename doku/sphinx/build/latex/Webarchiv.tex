% Generated by Sphinx.
\def\sphinxdocclass{report}
\documentclass[letterpaper,10pt,english]{sphinxmanual}
\usepackage[utf8]{inputenc}
\DeclareUnicodeCharacter{00A0}{\nobreakspace}
\usepackage[T1]{fontenc}
\usepackage{babel}
\usepackage{times}
\usepackage[Bjarne]{fncychap}
\usepackage{longtable}
\usepackage{sphinx}
\usepackage{multirow}


\title{Webarchiv Documentation}
\date{June 16, 2012}
\release{1.0}
\author{Christopher Pahl, Christoph Piechula}
\newcommand{\sphinxlogo}{}
\renewcommand{\releasename}{Release}
\makeindex

\makeatletter
\def\PYG@reset{\let\PYG@it=\relax \let\PYG@bf=\relax%
    \let\PYG@ul=\relax \let\PYG@tc=\relax%
    \let\PYG@bc=\relax \let\PYG@ff=\relax}
\def\PYG@tok#1{\csname PYG@tok@#1\endcsname}
\def\PYG@toks#1+{\ifx\relax#1\empty\else%
    \PYG@tok{#1}\expandafter\PYG@toks\fi}
\def\PYG@do#1{\PYG@bc{\PYG@tc{\PYG@ul{%
    \PYG@it{\PYG@bf{\PYG@ff{#1}}}}}}}
\def\PYG#1#2{\PYG@reset\PYG@toks#1+\relax+\PYG@do{#2}}

\expandafter\def\csname PYG@tok@gd\endcsname{\def\PYG@tc##1{\textcolor[rgb]{0.63,0.00,0.00}{##1}}}
\expandafter\def\csname PYG@tok@gu\endcsname{\let\PYG@bf=\textbf\def\PYG@tc##1{\textcolor[rgb]{0.50,0.00,0.50}{##1}}}
\expandafter\def\csname PYG@tok@gt\endcsname{\def\PYG@tc##1{\textcolor[rgb]{0.00,0.25,0.82}{##1}}}
\expandafter\def\csname PYG@tok@gs\endcsname{\let\PYG@bf=\textbf}
\expandafter\def\csname PYG@tok@gr\endcsname{\def\PYG@tc##1{\textcolor[rgb]{1.00,0.00,0.00}{##1}}}
\expandafter\def\csname PYG@tok@cm\endcsname{\let\PYG@it=\textit\def\PYG@tc##1{\textcolor[rgb]{0.25,0.50,0.56}{##1}}}
\expandafter\def\csname PYG@tok@vg\endcsname{\def\PYG@tc##1{\textcolor[rgb]{0.73,0.38,0.84}{##1}}}
\expandafter\def\csname PYG@tok@m\endcsname{\def\PYG@tc##1{\textcolor[rgb]{0.13,0.50,0.31}{##1}}}
\expandafter\def\csname PYG@tok@mh\endcsname{\def\PYG@tc##1{\textcolor[rgb]{0.13,0.50,0.31}{##1}}}
\expandafter\def\csname PYG@tok@cs\endcsname{\def\PYG@tc##1{\textcolor[rgb]{0.25,0.50,0.56}{##1}}\def\PYG@bc##1{\setlength{\fboxsep}{0pt}\colorbox[rgb]{1.00,0.94,0.94}{\strut ##1}}}
\expandafter\def\csname PYG@tok@ge\endcsname{\let\PYG@it=\textit}
\expandafter\def\csname PYG@tok@vc\endcsname{\def\PYG@tc##1{\textcolor[rgb]{0.73,0.38,0.84}{##1}}}
\expandafter\def\csname PYG@tok@il\endcsname{\def\PYG@tc##1{\textcolor[rgb]{0.13,0.50,0.31}{##1}}}
\expandafter\def\csname PYG@tok@go\endcsname{\def\PYG@tc##1{\textcolor[rgb]{0.19,0.19,0.19}{##1}}}
\expandafter\def\csname PYG@tok@cp\endcsname{\def\PYG@tc##1{\textcolor[rgb]{0.00,0.44,0.13}{##1}}}
\expandafter\def\csname PYG@tok@gi\endcsname{\def\PYG@tc##1{\textcolor[rgb]{0.00,0.63,0.00}{##1}}}
\expandafter\def\csname PYG@tok@gh\endcsname{\let\PYG@bf=\textbf\def\PYG@tc##1{\textcolor[rgb]{0.00,0.00,0.50}{##1}}}
\expandafter\def\csname PYG@tok@ni\endcsname{\let\PYG@bf=\textbf\def\PYG@tc##1{\textcolor[rgb]{0.84,0.33,0.22}{##1}}}
\expandafter\def\csname PYG@tok@nl\endcsname{\let\PYG@bf=\textbf\def\PYG@tc##1{\textcolor[rgb]{0.00,0.13,0.44}{##1}}}
\expandafter\def\csname PYG@tok@nn\endcsname{\let\PYG@bf=\textbf\def\PYG@tc##1{\textcolor[rgb]{0.05,0.52,0.71}{##1}}}
\expandafter\def\csname PYG@tok@no\endcsname{\def\PYG@tc##1{\textcolor[rgb]{0.38,0.68,0.84}{##1}}}
\expandafter\def\csname PYG@tok@na\endcsname{\def\PYG@tc##1{\textcolor[rgb]{0.25,0.44,0.63}{##1}}}
\expandafter\def\csname PYG@tok@nb\endcsname{\def\PYG@tc##1{\textcolor[rgb]{0.00,0.44,0.13}{##1}}}
\expandafter\def\csname PYG@tok@nc\endcsname{\let\PYG@bf=\textbf\def\PYG@tc##1{\textcolor[rgb]{0.05,0.52,0.71}{##1}}}
\expandafter\def\csname PYG@tok@nd\endcsname{\let\PYG@bf=\textbf\def\PYG@tc##1{\textcolor[rgb]{0.33,0.33,0.33}{##1}}}
\expandafter\def\csname PYG@tok@ne\endcsname{\def\PYG@tc##1{\textcolor[rgb]{0.00,0.44,0.13}{##1}}}
\expandafter\def\csname PYG@tok@nf\endcsname{\def\PYG@tc##1{\textcolor[rgb]{0.02,0.16,0.49}{##1}}}
\expandafter\def\csname PYG@tok@si\endcsname{\let\PYG@it=\textit\def\PYG@tc##1{\textcolor[rgb]{0.44,0.63,0.82}{##1}}}
\expandafter\def\csname PYG@tok@s2\endcsname{\def\PYG@tc##1{\textcolor[rgb]{0.25,0.44,0.63}{##1}}}
\expandafter\def\csname PYG@tok@vi\endcsname{\def\PYG@tc##1{\textcolor[rgb]{0.73,0.38,0.84}{##1}}}
\expandafter\def\csname PYG@tok@nt\endcsname{\let\PYG@bf=\textbf\def\PYG@tc##1{\textcolor[rgb]{0.02,0.16,0.45}{##1}}}
\expandafter\def\csname PYG@tok@nv\endcsname{\def\PYG@tc##1{\textcolor[rgb]{0.73,0.38,0.84}{##1}}}
\expandafter\def\csname PYG@tok@s1\endcsname{\def\PYG@tc##1{\textcolor[rgb]{0.25,0.44,0.63}{##1}}}
\expandafter\def\csname PYG@tok@gp\endcsname{\let\PYG@bf=\textbf\def\PYG@tc##1{\textcolor[rgb]{0.78,0.36,0.04}{##1}}}
\expandafter\def\csname PYG@tok@sh\endcsname{\def\PYG@tc##1{\textcolor[rgb]{0.25,0.44,0.63}{##1}}}
\expandafter\def\csname PYG@tok@ow\endcsname{\let\PYG@bf=\textbf\def\PYG@tc##1{\textcolor[rgb]{0.00,0.44,0.13}{##1}}}
\expandafter\def\csname PYG@tok@sx\endcsname{\def\PYG@tc##1{\textcolor[rgb]{0.78,0.36,0.04}{##1}}}
\expandafter\def\csname PYG@tok@bp\endcsname{\def\PYG@tc##1{\textcolor[rgb]{0.00,0.44,0.13}{##1}}}
\expandafter\def\csname PYG@tok@c1\endcsname{\let\PYG@it=\textit\def\PYG@tc##1{\textcolor[rgb]{0.25,0.50,0.56}{##1}}}
\expandafter\def\csname PYG@tok@kc\endcsname{\let\PYG@bf=\textbf\def\PYG@tc##1{\textcolor[rgb]{0.00,0.44,0.13}{##1}}}
\expandafter\def\csname PYG@tok@c\endcsname{\let\PYG@it=\textit\def\PYG@tc##1{\textcolor[rgb]{0.25,0.50,0.56}{##1}}}
\expandafter\def\csname PYG@tok@mf\endcsname{\def\PYG@tc##1{\textcolor[rgb]{0.13,0.50,0.31}{##1}}}
\expandafter\def\csname PYG@tok@err\endcsname{\def\PYG@bc##1{\setlength{\fboxsep}{0pt}\fcolorbox[rgb]{1.00,0.00,0.00}{1,1,1}{\strut ##1}}}
\expandafter\def\csname PYG@tok@kd\endcsname{\let\PYG@bf=\textbf\def\PYG@tc##1{\textcolor[rgb]{0.00,0.44,0.13}{##1}}}
\expandafter\def\csname PYG@tok@ss\endcsname{\def\PYG@tc##1{\textcolor[rgb]{0.32,0.47,0.09}{##1}}}
\expandafter\def\csname PYG@tok@sr\endcsname{\def\PYG@tc##1{\textcolor[rgb]{0.14,0.33,0.53}{##1}}}
\expandafter\def\csname PYG@tok@mo\endcsname{\def\PYG@tc##1{\textcolor[rgb]{0.13,0.50,0.31}{##1}}}
\expandafter\def\csname PYG@tok@mi\endcsname{\def\PYG@tc##1{\textcolor[rgb]{0.13,0.50,0.31}{##1}}}
\expandafter\def\csname PYG@tok@kn\endcsname{\let\PYG@bf=\textbf\def\PYG@tc##1{\textcolor[rgb]{0.00,0.44,0.13}{##1}}}
\expandafter\def\csname PYG@tok@o\endcsname{\def\PYG@tc##1{\textcolor[rgb]{0.40,0.40,0.40}{##1}}}
\expandafter\def\csname PYG@tok@kr\endcsname{\let\PYG@bf=\textbf\def\PYG@tc##1{\textcolor[rgb]{0.00,0.44,0.13}{##1}}}
\expandafter\def\csname PYG@tok@s\endcsname{\def\PYG@tc##1{\textcolor[rgb]{0.25,0.44,0.63}{##1}}}
\expandafter\def\csname PYG@tok@kp\endcsname{\def\PYG@tc##1{\textcolor[rgb]{0.00,0.44,0.13}{##1}}}
\expandafter\def\csname PYG@tok@w\endcsname{\def\PYG@tc##1{\textcolor[rgb]{0.73,0.73,0.73}{##1}}}
\expandafter\def\csname PYG@tok@kt\endcsname{\def\PYG@tc##1{\textcolor[rgb]{0.56,0.13,0.00}{##1}}}
\expandafter\def\csname PYG@tok@sc\endcsname{\def\PYG@tc##1{\textcolor[rgb]{0.25,0.44,0.63}{##1}}}
\expandafter\def\csname PYG@tok@sb\endcsname{\def\PYG@tc##1{\textcolor[rgb]{0.25,0.44,0.63}{##1}}}
\expandafter\def\csname PYG@tok@k\endcsname{\let\PYG@bf=\textbf\def\PYG@tc##1{\textcolor[rgb]{0.00,0.44,0.13}{##1}}}
\expandafter\def\csname PYG@tok@se\endcsname{\let\PYG@bf=\textbf\def\PYG@tc##1{\textcolor[rgb]{0.25,0.44,0.63}{##1}}}
\expandafter\def\csname PYG@tok@sd\endcsname{\let\PYG@it=\textit\def\PYG@tc##1{\textcolor[rgb]{0.25,0.44,0.63}{##1}}}

\def\PYGZbs{\char`\\}
\def\PYGZus{\char`\_}
\def\PYGZob{\char`\{}
\def\PYGZcb{\char`\}}
\def\PYGZca{\char`\^}
\def\PYGZam{\char`\&}
\def\PYGZlt{\char`\<}
\def\PYGZgt{\char`\>}
\def\PYGZsh{\char`\#}
\def\PYGZpc{\char`\%}
\def\PYGZdl{\char`\$}
\def\PYGZti{\char`\~}
% for compatibility with earlier versions
\def\PYGZat{@}
\def\PYGZlb{[}
\def\PYGZrb{]}
\makeatother

\begin{document}

\maketitle
\tableofcontents
\phantomsection\label{index::doc}


Contents:


\chapter{Commandlineinterface}
\label{cli:commandlineinterface}\label{cli::doc}\label{cli:welcome-to-webarchiv-s-documentation}

\section{Quick-Reference}
\label{cli:quick-reference}
\begin{Verbatim}[commandchars=\\\{\}]
Usage:
  archive.py [--loglevel=\textless{}severity\textgreater{}] init [\textless{}path\textgreater{}]
  archive.py [--loglevel=\textless{}severity\textgreater{}] crawler
  archive.py [--loglevel=\textless{}severity\textgreater{}] javadapter
  archive.py [--loglevel=\textless{}severity\textgreater{}] db (--rebuild\textbar{}--remove)
  archive.py [--loglevel=\textless{}severity\textgreater{}] repair
  archive.py config (--get=\textless{}confurl\textgreater{}\textbar{}--set=\textless{}confurl\textgreater{}\textless{}arg\textgreater{})
  archive.py -h \textbar{} --help
  archive.py --version

General Options:
  -h --help                Show this screen.
  --version                Show version.
  --loglevel=\textless{}loglevel\textgreater{}    Set the loglevel to any of debug, info, warning, error, critical.

DB Options:
  --rebuild                Rebuild Databse completely from XML Data.
  --remove                 Remove the Database completely.

Config Options:
  --set=\textless{}confurl\textgreater{}\textless{}value\textgreater{}   Set a Value in the config permanently.
  --get=\textless{}confurl\textgreater{}          Acquire a Value in the config by it's url.
\end{Verbatim}


\section{Additional Notes}
\label{cli:additional-notes}\begin{itemize}
\item {} 
The Commandline interfaces relies on submodules like \emph{crawler}, \emph{config} ...

\item {} 
Submodules may have own options

\item {} 
Before stating the submodule common options may be set (e.g. \emph{--loglevel})

\item {} 
The submodules \emph{javadapter} and \emph{crawler} start a special shell

\item {} 
In order to locate the config you either have to pass it explicitly, or your current working directory is at the archive root.

\end{itemize}


\section{Implementation}
\label{cli:implementation}\index{Cli (class in cli.cmdparser)}

\begin{fulllineitems}
\phantomsection\label{cli:cli.cmdparser.Cli}\pysigline{\strong{class }\code{cli.cmdparser.}\bfcode{Cli}}
Archive commandline intepreter
\index{cmd\_loop() (cli.cmdparser.Cli method)}

\begin{fulllineitems}
\phantomsection\label{cli:cli.cmdparser.Cli.cmd_loop}\pysiglinewithargsret{\bfcode{cmd\_loop}}{\emph{shell}, \emph{i}, \emph{cv}}{}
The cmdloop runs in a seperate thread.

\end{fulllineitems}

\index{handle\_config() (cli.cmdparser.Cli method)}

\begin{fulllineitems}
\phantomsection\label{cli:cli.cmdparser.Cli.handle_config}\pysiglinewithargsret{\bfcode{handle\_config}}{}{}
Invokes Config Handler operations

\end{fulllineitems}

\index{handle\_crawler() (cli.cmdparser.Cli method)}

\begin{fulllineitems}
\phantomsection\label{cli:cli.cmdparser.Cli.handle_crawler}\pysiglinewithargsret{\bfcode{handle\_crawler}}{}{}
Starts and controls crawler commandline

\end{fulllineitems}

\index{handle\_db() (cli.cmdparser.Cli method)}

\begin{fulllineitems}
\phantomsection\label{cli:cli.cmdparser.Cli.handle_db}\pysiglinewithargsret{\bfcode{handle\_db}}{}{}
Handle ``db'' submodule

\end{fulllineitems}

\index{handle\_init() (cli.cmdparser.Cli method)}

\begin{fulllineitems}
\phantomsection\label{cli:cli.cmdparser.Cli.handle_init}\pysiglinewithargsret{\bfcode{handle\_init}}{}{}
Initializes archive paths

\end{fulllineitems}

\index{handle\_javadapter() (cli.cmdparser.Cli method)}

\begin{fulllineitems}
\phantomsection\label{cli:cli.cmdparser.Cli.handle_javadapter}\pysiglinewithargsret{\bfcode{handle\_javadapter}}{}{}
Starts javadapter commandline

\end{fulllineitems}

\index{handle\_repair() (cli.cmdparser.Cli method)}

\begin{fulllineitems}
\phantomsection\label{cli:cli.cmdparser.Cli.handle_repair}\pysiglinewithargsret{\bfcode{handle\_repair}}{}{}
Invokes archive rapair tool

\end{fulllineitems}


\end{fulllineitems}



\chapter{Python to Java Interface (aka Javadapter)}
\label{javadapter:python-to-java-interface-aka-javadapter}\label{javadapter::doc}

\section{Description}
\label{javadapter:description}
The Javadapter is a simple TCPServer that will listen by default on port \code{42421} on \code{localhost}.
One may connect to this server and send one of the commands below.
On success the server will send a reponse that is terminated with \code{OK}, otherwise
\code{ACK Some Error Description.} is send.

The Server may be started via:

\begin{Verbatim}[commandchars=\\\{\}]
\PYG{n+nv}{\PYGZdl{} }archive.py javadapter --start
\PYG{c}{\PYGZsh{} This will enter a special shell.}
\PYG{c}{\PYGZsh{} Use {}`help{}` to see what you can do there}
\end{Verbatim}


\section{List of commands}
\label{javadapter:list-of-commands}\begin{description}
\item[{\textbf{lock}}] \leavevmode\begin{description}
\item[{\emph{description:}}] \leavevmode
Lock a domain and wait to a maximal time
of 5 minutes, will return a timeout then

\item[{\emph{usage:}}] \leavevmode
\code{lock {[}domain{]}}
\begin{itemize}
\item {} 
domain is e.g. www.heise.de

\item {} 
Returns nothing

\end{itemize}

\item[{\emph{examples:}}] \leavevmode
\begin{Verbatim}[commandchars=\\\{\}]
\PYG{n+nv}{\PYGZdl{} }lock www.heise.de
OK
\PYG{n+nv}{\PYGZdl{} }lock www.heise.de
\PYG{o}{(}... will timeout after 5 mins ...\PYG{o}{)}
ACK Timeout occured.
\PYG{n+nv}{\PYGZdl{} }unlock www.heise.de
OK
\PYG{n+nv}{\PYGZdl{} }lock www.heise.de
OK
\end{Verbatim}

\end{description}

\end{description}


\bigskip\hrule{}\bigskip

\begin{description}
\item[{\textbf{try\_lock}}] \leavevmode\begin{description}
\item[{\emph{description:}}] \leavevmode
As \code{lock}, but return immediately with
\code{ACK Already locked.} if already locked previously.

\item[{\emph{usage:}}] \leavevmode
\code{try\_lock {[}domain{]}}
\begin{itemize}
\item {} 
domain is e.g. www.heise.de

\item {} 
Returns nothing

\end{itemize}

\item[{\emph{examples:}}] \leavevmode
\begin{Verbatim}[commandchars=\\\{\}]
\PYG{n+nv}{\PYGZdl{} }try\PYGZus{}lock www.heise.de
OK
\PYG{n+nv}{\PYGZdl{} }try\PYGZus{}lock www.heise.de
ACK Already locked.
\end{Verbatim}

\end{description}

\end{description}


\bigskip\hrule{}\bigskip

\begin{description}
\item[{\textbf{unlock}}] \leavevmode\begin{description}
\item[{\emph{description:}}] \leavevmode
Unlock a previous lock

\item[{\emph{usage:}}] \leavevmode
\code{unlock {[}domain{]}}
\begin{itemize}
\item {} 
domain is e.g. www.heise.de

\item {} 
Returns nothing

\end{itemize}

\item[{\emph{examples:}}] \leavevmode
\begin{Verbatim}[commandchars=\\\{\}]
\PYG{n+nv}{\PYGZdl{} }unlock www.heise.de
OK
\PYG{n+nv}{\PYGZdl{} }unlock www.youporn.com
ACK Invalid Domain.
\end{Verbatim}

\end{description}

\end{description}


\bigskip\hrule{}\bigskip

\begin{description}
\item[{\textbf{checkout}}] \leavevmode\begin{description}
\item[{\emph{description:}}] \leavevmode
Checkout a certain branch (usually a commitTag or \code{master})
You do not need to manually set a lock for this.

\item[{\emph{usage:}}] \leavevmode
\code{checkout {[}domain{]} \{branch\_name\}}
\begin{itemize}
\item {} 
domain is e.g. www.heise.de

\item {} 
branch\_name the entity to checkout, if omitted only
the path is returned (if valid) and no git work is done.

\item {} 
Returns: The Path to the checkout'd domain

\end{itemize}

\item[{\emph{warning:}}] \leavevmode
\textbf{Note:} You should always checkout \code{master} when done!

\item[{\emph{examples:}}] \leavevmode
\begin{Verbatim}[commandchars=\\\{\}]
\PYG{n+nv}{\PYGZdl{} }checkout www.hack.org 2012H06H15T19C08C15
/tmp/archive/content/www.hack.org
OK
\PYG{n+nv}{\PYGZdl{} }checkout www.youporn.com
ACK Invalid Domain.
\PYG{n+nv}{\PYGZdl{} }checkout www.hack.org no\PYGZus{}branch\PYGZus{}name
ACK checkout returned 1
\end{Verbatim}

\end{description}

\end{description}


\bigskip\hrule{}\bigskip

\begin{description}
\item[{\textbf{commit}}] \leavevmode\begin{description}
\item[{\emph{description:}}] \leavevmode
Make a commit on a certain domain.

\item[{\emph{usage:}}] \leavevmode
\code{commit {[}domain{]} \{message\}}
\begin{itemize}
\item {} 
domain is e.g. www.heise.de

\item {} 
message is the commit message (optional, \code{edit} by default)

\item {} 
Returns nothing

\end{itemize}

\item[{\emph{examples:}}] \leavevmode
\begin{Verbatim}[commandchars=\\\{\}]
\PYG{n+nv}{\PYGZdl{} }commit www.hack.org HelloWorld
ACK commit returned 1
\PYG{c}{\PYGZsh{} Uh-Oh, nothing to commit - add some content manually}
user@arc \PYG{n+nv}{\PYGZdl{} }touch /tmp/archive/content/www.hack.org/new\PYGZus{}file
\PYG{c}{\PYGZsh{} Now commiting works:}
\PYG{n+nv}{\PYGZdl{} }commit www.hack.org
OK
\end{Verbatim}

\end{description}

\end{description}


\bigskip\hrule{}\bigskip

\begin{description}
\item[{\textbf{list\_commits:}}] \leavevmode\begin{description}
\item[{\emph{description:}}] \leavevmode
List all commits on a certain domain and its current branch.

\item[{\emph{usage:}}] \leavevmode
\code{list\_commits {[}domain{]}}
\begin{itemize}
\item {} 
domain is e.g. www.heise.de

\item {} 
Returns a newline seperated list of commithashes

\end{itemize}

\item[{\emph{examples:}}] \leavevmode
\begin{Verbatim}[commandchars=\\\{\}]
\PYG{n+nv}{\PYGZdl{} }list\PYGZus{}commits www.hack.org
6309b01f5b04b4e60c19f5dd147f935f40d94840
942f9a1da172592228d22ca638dd3f5ae583d285
OK
\end{Verbatim}

\end{description}

\end{description}


\bigskip\hrule{}\bigskip

\begin{description}
\item[{\textbf{list\_branches:}}] \leavevmode\begin{description}
\item[{\emph{description:}}] \leavevmode
List all branches on a certain domain.

\item[{\emph{usage:}}] \leavevmode
\code{list\_branches {[}domain{]}}
\begin{itemize}
\item {} 
domain is e.g. www.heise.de

\item {} 
Returns a newline seperated list of branchnames

\end{itemize}

\item[{\emph{examples:}}] \leavevmode
\begin{Verbatim}[commandchars=\\\{\}]
\PYG{n+nv}{\PYGZdl{} }list\PYGZus{}branches www.hack.org
2012H06H13T23C02C18
2012H06H15T19C07C46
2012H06H15T19C08C15
2012H06H15T21C57C35
2012H06H15T21C57C43
\PYG{c}{\PYGZsh{} (..snip..)}
OK
\end{Verbatim}

\end{description}

\end{description}


\section{Implementation}
\label{javadapter:implementation}
\textbf{Actual function to start the server:}
\index{start() (in module javadapter.server)}

\begin{fulllineitems}
\phantomsection\label{javadapter:javadapter.server.start}\pysiglinewithargsret{\code{javadapter.server.}\bfcode{start}}{\emph{host='localhost'}, \emph{port=42421}}{}
Start the Javadapter server, and exit once done
\begin{quote}\begin{description}
\item[{Host }] \leavevmode
the host to start the server on (does anythinh but localhost work?)

\item[{Port }] \leavevmode
the port on which the server listens on

\item[{Returns}] \leavevmode
a server, on which shutdown() can be called

\end{description}\end{quote}

\end{fulllineitems}



\bigskip\hrule{}\bigskip


\textbf{Convienience class to show a servershell:}
\index{ServerShell (class in javadapter.server)}

\begin{fulllineitems}
\phantomsection\label{javadapter:javadapter.server.ServerShell}\pysiglinewithargsret{\strong{class }\code{javadapter.server.}\bfcode{ServerShell}}{\emph{host='localhost'}, \emph{port=`42421'}, \emph{server\_instance=None}}{}
Command shell to manage javadapter
\index{do\_EOF() (javadapter.server.ServerShell method)}

\begin{fulllineitems}
\phantomsection\label{javadapter:javadapter.server.ServerShell.do_EOF}\pysiglinewithargsret{\bfcode{do\_EOF}}{\emph{arg}}{}
Shortcut for quit (Press CTRL+D)

\end{fulllineitems}

\index{do\_quit() (javadapter.server.ServerShell method)}

\begin{fulllineitems}
\phantomsection\label{javadapter:javadapter.server.ServerShell.do_quit}\pysiglinewithargsret{\bfcode{do\_quit}}{\emph{arg}}{}
Quits the server

\end{fulllineitems}

\index{do\_start() (javadapter.server.ServerShell method)}

\begin{fulllineitems}
\phantomsection\label{javadapter:javadapter.server.ServerShell.do_start}\pysiglinewithargsret{\bfcode{do\_start}}{\emph{arg}}{}
Start a server if not already active

\end{fulllineitems}

\index{do\_status() (javadapter.server.ServerShell method)}

\begin{fulllineitems}
\phantomsection\label{javadapter:javadapter.server.ServerShell.do_status}\pysiglinewithargsret{\bfcode{do\_status}}{\emph{arg}}{}
Print current status of the Server

\end{fulllineitems}

\index{do\_stop() (javadapter.server.ServerShell method)}

\begin{fulllineitems}
\phantomsection\label{javadapter:javadapter.server.ServerShell.do_stop}\pysiglinewithargsret{\bfcode{do\_stop}}{\emph{arg}}{}
Stop a running Server

\end{fulllineitems}


\end{fulllineitems}



\chapter{Git-Handling}
\label{git:git-handling}\label{git::doc}

\section{Overview}
\label{git:overview}
\textbf{Initialization:}

\begin{Verbatim}[commandchars=\\\{\}]
\PYG{n+nv}{\PYGZdl{} }git init
\PYG{n+nv}{\PYGZdl{} }git checkout -b empty
\PYG{c}{\PYGZsh{} At least one commit is needed for a valid branch}
\PYG{n+nv}{\PYGZdl{} }\PYG{n+nb}{echo} \PYG{l+s+s1}{'This is Empty'} \PYGZgt{} EMPTY
\PYG{n+nv}{\PYGZdl{} }git add EMPTY
\PYG{n+nv}{\PYGZdl{} }git commit -a -m \PYG{l+s+s1}{'Init'}
\PYG{c}{\PYGZsh{} master will be used to track}
\PYG{c}{\PYGZsh{} the most recent branch}
\end{Verbatim}

\textbf{Synchronization:}

\begin{Verbatim}[commandchars=\\\{\}]
... lock ...
\PYG{c}{\PYGZsh{} Gehe zum leeren stand zurück,}
\PYG{c}{\PYGZsh{} da sonst der neue branch die history}
\PYG{c}{\PYGZsh{} des aktuellen erbt}
\PYG{n+nv}{\PYGZdl{} }git checkout empty
\PYG{c}{\PYGZsh{} Neuer branch mit ehem. Tagnamen}
\PYG{n+nv}{\PYGZdl{} }git checkout -b 24052012T1232
... rsync ins Archiv ...
\PYG{n+nv}{\PYGZdl{} }git add .
\PYG{n+nv}{\PYGZdl{} }git commit -am \PYG{l+s+s1}{'Seite xyz.com wurde gekrault'}
\PYG{c}{\PYGZsh{} Nun ist 'master' mit dem aktuellsten Stand identisch}
\PYG{n+nv}{\PYGZdl{} }git branch -d master
\PYG{n+nv}{\PYGZdl{} }git checkout -b master
... unlock ...
\end{Verbatim}

\textbf{Reading/Writing on most recent data:}

\begin{Verbatim}[commandchars=\\\{\}]
\PYG{c}{\PYGZsh{} Not git-work required}
... lock ...
... \PYG{n+nb}{read} ...
... unlock ...
\end{Verbatim}

\begin{Verbatim}[commandchars=\\\{\}]
\PYG{c}{\PYGZsh{} Lesen / Editieren von alten Ständen}
\PYG{c}{\PYGZsh{} Hierfür muss das Datum des alten Standes gegeben sein}
\PYG{c}{\PYGZsh{} -f falls jemand unerlaubt änderungen gemacht hat}
... lock ...
\PYG{n+nv}{\PYGZdl{} }git checkout -f old\PYGZus{}date
... lesen / schreiben ...
\PYG{c}{\PYGZsh{} Im Falle von schreiben:}
\PYG{n+nv}{\PYGZdl{} }git add .
\PYG{n+nv}{\PYGZdl{} }git commit -am \PYG{l+s+s1}{'Edited old Kraul'}
\PYG{c}{\PYGZsh{} Der Kopf des neuen branches zeigt nun auf den neuen commit}
\PYG{n+nv}{\PYGZdl{} }git checkout master
... unlock ...
\end{Verbatim}

Rough schema as ASCII-Art:

\begin{Verbatim}[commandchars=\\\{\}]
          -- Kraul1 -\PYGZgt{} edit \PYGZlt{}- branch \PYG{l+s+s1}{'03052012T1232'}
        /
Init -- ---- Kraul2 \PYGZlt{}- branch \PYG{l+s+s1}{'15052012T1232'}
\textbar{}       \PYG{l+s+se}{\PYGZbs{}}
\textbar{}         -- Kraul3 \PYGZlt{}- branch \PYG{l+s+s1}{'24052012T1232'} \PYGZlt{}- branch \PYG{l+s+s1}{'master'}
\textbar{}
\textbar{}
\PYG{l+s+se}{\PYGZbs{}-}\PYGZgt{} branch \PYG{l+s+s1}{'empty'}
\end{Verbatim}

Previously, with the \code{tag} approach:

\begin{Verbatim}[commandchars=\\\{\}]
Kraul1 -\PYGZgt{} Kraul2 -\PYGZgt{} Kraul3 -\PYGZgt{} Kraul4 \PYGZlt{}- branch \PYG{l+s+s1}{'master'}
\textbar{}         \textbar{}         \textbar{}         \textbar{}
\textbar{}         \textbar{}         \textbar{}          \PYG{l+s+se}{\PYGZbs{}}
\textbar{}         \textbar{}         \PYG{l+s+se}{\PYGZbs{} }           -- Tag 04
\textbar{}         \PYG{l+s+se}{\PYGZbs{} }          -- Tag 03
\PYG{l+s+se}{\PYGZbs{} }          -- Tag 02
  -- Tag 01


\PYG{c}{\PYGZsh{} Vorteile:}
\PYG{c}{\PYGZsh{} - Alte Stände editierbar}
\PYG{c}{\PYGZsh{} - Überprüfung ob aktuell fällt weg}
\PYG{c}{\PYGZsh{} - (Seltsamerweise) weniger Platzverbrauch}

\PYG{c}{\PYGZsh{} Nachteile:}
\PYG{c}{\PYGZsh{} - Alte Stände auschecken (vermutlich) langsamer als normale git tag checkouts}
\PYG{c}{\PYGZsh{} - Traversieren über alle Stände (für DB Recover wird etwas schwieriger) - aber ist möglich.}
\end{Verbatim}


\section{Implementation}
\label{git:implementation}\label{git:module-crawler.git}\index{crawler.git (module)}
Wrapper for Git

This is highly simplified, and may be replaced
by a faster, native implementation using Dunwhich.
But that's not on the plan due our limited time.

Git commands (init e.g.) are tailored for use
in this archive, less for general use.
\index{Git (class in crawler.git)}

\begin{fulllineitems}
\phantomsection\label{git:crawler.git.Git}\pysiglinewithargsret{\strong{class }\code{crawler.git.}\bfcode{Git}}{\emph{domain}}{}
A (overly-simple) Wrapper for the git binary
\index{branch() (crawler.git.Git method)}

\begin{fulllineitems}
\phantomsection\label{git:crawler.git.Git.branch}\pysiglinewithargsret{\bfcode{branch}}{\emph{branch\_name='empty'}}{}
create a new named branch
\begin{quote}\begin{description}
\item[{Branch\_name }] \leavevmode
the name of the new branch, may not exist yet

\item[{Returns}] \leavevmode
0 on success, another rcode on failure

\end{description}\end{quote}

\end{fulllineitems}

\index{checkout() (crawler.git.Git method)}

\begin{fulllineitems}
\phantomsection\label{git:crawler.git.Git.checkout}\pysiglinewithargsret{\bfcode{checkout}}{\emph{target='master'}}{}
checkout a certain point (tag, branch or commit)
\begin{quote}\begin{description}
\item[{Target }] \leavevmode
the target to visit

\item[{Returns}] \leavevmode
0 on success, another rcode on failure

\end{description}\end{quote}

\end{fulllineitems}

\index{commit() (crawler.git.Git method)}

\begin{fulllineitems}
\phantomsection\label{git:crawler.git.Git.commit}\pysiglinewithargsret{\bfcode{commit}}{\emph{message='edit'}}{}
commit any changes made

git add . and git commit -am \textless{}message\textgreater{} is done
\begin{quote}\begin{description}
\item[{Message }] \leavevmode
The commit message

\item[{Returns}] \leavevmode
0 on success, another rcode on failure

\end{description}\end{quote}

\end{fulllineitems}

\index{convert\_branch\_name() (crawler.git.Git class method)}

\begin{fulllineitems}
\phantomsection\label{git:crawler.git.Git.convert_branch_name}\pysiglinewithargsret{\strong{classmethod }\bfcode{convert\_branch\_name}}{\emph{date\_string}}{}
Convert a datestring suitably to a branch name

Git does not allow special characters such as : or -
in branchnames for whatever reason
\begin{quote}\begin{description}
\item[{Parameters}] \leavevmode
\textbf{date\_string} -- the string to convert

\item[{Returns}] \leavevmode
the new, converted string

\end{description}\end{quote}

\end{fulllineitems}

\index{domain (crawler.git.Git attribute)}

\begin{fulllineitems}
\phantomsection\label{git:crawler.git.Git.domain}\pysigline{\bfcode{domain}}
Return the domain, to which this wrapper belongs

\end{fulllineitems}

\index{init() (crawler.git.Git method)}

\begin{fulllineitems}
\phantomsection\label{git:crawler.git.Git.init}\pysiglinewithargsret{\bfcode{init}}{}{}
Create a new archive at specified domain path

The target directory does not need to exit yet
\begin{quote}\begin{description}
\item[{Returns}] \leavevmode
0 on success, another rcode on failure

\end{description}\end{quote}

\end{fulllineitems}

\index{list\_branches() (crawler.git.Git method)}

\begin{fulllineitems}
\phantomsection\label{git:crawler.git.Git.list_branches}\pysiglinewithargsret{\bfcode{list\_branches}}{}{}
List all branches in this repo, which conform to the `date'-regex.

This means, Empty and master branch are not mentioned. If you want
to checkout those, just checkout `empty' or `master'
\begin{quote}\begin{description}
\item[{Returns}] \leavevmode
a list of branchestrings

\end{description}\end{quote}

\end{fulllineitems}

\index{list\_commits() (crawler.git.Git method)}

\begin{fulllineitems}
\phantomsection\label{git:crawler.git.Git.list_commits}\pysiglinewithargsret{\bfcode{list\_commits}}{}{}
List all commits in this repo and branch
\begin{quote}\begin{description}
\item[{Returns}] \leavevmode
a list of commithashestrings

\end{description}\end{quote}

\end{fulllineitems}

\index{recreate\_master() (crawler.git.Git method)}

\begin{fulllineitems}
\phantomsection\label{git:crawler.git.Git.recreate_master}\pysiglinewithargsret{\bfcode{recreate\_master}}{}{}
A very special helper.

It deletes the current master branch,
and recreates it. So, the master always points
to the most recently created branch

\end{fulllineitems}


\end{fulllineitems}



\chapter{Database Generation}
\label{dbgen:database-generation}\label{dbgen::doc}

\section{Overview}
\label{dbgen:overview}
On the very end of every run of the crawler an update is done on the database, by
iterating over all data in the internal metadata-list and builiding SQL Statements from this.

Insertion, for every:
\begin{enumerate}
\item {} 
... domain a new row is inserted to the \emph{domain} table. (Already existent domains are ignored)

\item {} 
... mimeType a new row is inserted to the \emph{mimeType} table.

\item {} 
... url and path a new row is inserted to the \emph{metaData} table.

\item {} 
... commit a new row is inserted to the \emph{commitTag} table, with a reference to the corresponding domain.

\item {} 
... new file committed to the archive a new row is inserted into the \emph{history} table.

\end{enumerate}

If a row with this data already exists it is ignored.


\section{Implementation}
\label{dbgen:implementation}
For Peformance-reasons only very simple insert-statements are used in combination
with as simple select statements, instead of insert-statements with sub-selects.


\bigskip\hrule{}\bigskip

\phantomsection\label{dbgen:module-crawler.dbgen}\index{crawler.dbgen (module)}
DBGenerator is capable of generating an sqlite database
from a list of metadictionaries.
\index{DBGenerator (class in crawler.dbgen)}

\begin{fulllineitems}
\phantomsection\label{dbgen:crawler.dbgen.DBGenerator}\pysiglinewithargsret{\strong{class }\code{crawler.dbgen.}\bfcode{DBGenerator}}{\emph{meta\_list=None}}{}
DBGenerator module
\index{batch() (crawler.dbgen.DBGenerator method)}

\begin{fulllineitems}
\phantomsection\label{dbgen:crawler.dbgen.DBGenerator.batch}\pysiglinewithargsret{\bfcode{batch}}{}{}
Start db creating procedure
\begin{quote}\begin{description}
\item[{Returns}] \leavevmode
a truthy value on success

\end{description}\end{quote}

\end{fulllineitems}

\index{close() (crawler.dbgen.DBGenerator method)}

\begin{fulllineitems}
\phantomsection\label{dbgen:crawler.dbgen.DBGenerator.close}\pysiglinewithargsret{\bfcode{close}}{}{}
Close connection and commit.

\end{fulllineitems}

\index{execute\_statement() (crawler.dbgen.DBGenerator method)}

\begin{fulllineitems}
\phantomsection\label{dbgen:crawler.dbgen.DBGenerator.execute_statement}\pysiglinewithargsret{\bfcode{execute\_statement}}{\emph{source\_name}, \emph{arglist=None}}{}
Exececute a previously loaded statement by name
\begin{quote}\begin{description}
\item[{Source\_name }] \leavevmode
Sourcename to execute (e.g. `create')

\item[{Arglist }] \leavevmode
You may pass an additional list of variable elements

\end{description}\end{quote}

\end{fulllineitems}

\index{insert\_history() (crawler.dbgen.DBGenerator method)}

\begin{fulllineitems}
\phantomsection\label{dbgen:crawler.dbgen.DBGenerator.insert_history}\pysiglinewithargsret{\bfcode{insert\_history}}{}{}
Fill history table

\end{fulllineitems}

\index{insert\_mdata\_ctag() (crawler.dbgen.DBGenerator method)}

\begin{fulllineitems}
\phantomsection\label{dbgen:crawler.dbgen.DBGenerator.insert_mdata_ctag}\pysiglinewithargsret{\bfcode{insert\_mdata\_ctag}}{}{}
Fill metadata and committag table

\end{fulllineitems}

\index{insert\_mime\_domain() (crawler.dbgen.DBGenerator method)}

\begin{fulllineitems}
\phantomsection\label{dbgen:crawler.dbgen.DBGenerator.insert_mime_domain}\pysiglinewithargsret{\bfcode{insert\_mime\_domain}}{}{}
Fill mimeType and domain table

\end{fulllineitems}

\index{load\_statements() (crawler.dbgen.DBGenerator method)}

\begin{fulllineitems}
\phantomsection\label{dbgen:crawler.dbgen.DBGenerator.load_statements}\pysiglinewithargsret{\bfcode{load\_statements}}{}{}
(Re-)Load Sql Files from Disk

This is already called in init
\begin{quote}\begin{description}
\item[{Returns}] \leavevmode
a dictionary with statements, indexed by name (e.g. `create')

\end{description}\end{quote}

\end{fulllineitems}

\index{select() (crawler.dbgen.DBGenerator method)}

\begin{fulllineitems}
\phantomsection\label{dbgen:crawler.dbgen.DBGenerator.select}\pysiglinewithargsret{\bfcode{select}}{\emph{table}, \emph{*columns}}{}
Internal helper for collecting data
\begin{quote}\begin{description}
\item[{Table }] \leavevmode
Table on which a SELECT shall be performed

\item[{Columns }] \leavevmode
a list of columns to select

\item[{Returns}] \leavevmode
A dictionary of column{[}0{]}: column{[}1:{]}

\end{description}\end{quote}

\end{fulllineitems}


\end{fulllineitems}



\chapter{Recovering of the Database}
\label{recover:recovering-of-the-database}\label{recover::doc}

\section{Strategies}
\label{recover:strategies}
Currently, there are two strategies to re-generate the Database:
\begin{description}
\item[{\textbf{Reading all XML Files:}}] \leavevmode
With this method the whole archive is traversed like this:
\begin{description}
\item[{Iterate over all domains.}] \leavevmode\begin{description}
\item[{Iterate over all branches of this domain (excluding \code{empty} branch)}] \leavevmode\begin{description}
\item[{Iterate over all commits (excluding \code{Init} commit)}] \leavevmode
Iterate over all XML Files in there and build metadata-dicts from them

\end{description}

\end{description}

\end{description}

From the generated metadata-list a new Database can be generated.

\emph{Advantages:}
\begin{itemize}
\item {} 
Works always, unless the archive is not totally broken

\item {} 
Also works for XML-Files that were modified somehow (also their baseattribs shouldn't)

\end{itemize}

\emph{Disadvantages:}
\begin{itemize}
\item {} 
May not be fast enough.

\end{itemize}

\end{description}

\textbf{Using Cached .pickle files:}
\begin{quote}

Instead of converting each XML costly to the internal representation, an object dump of the metadata-list
is written to \code{/archive-root/pickle\_cache/} on each crawl-run. If a recover is desired all of these
\emph{pickled} lists are joined, and the DB is regnerated.

\begin{Verbatim}[commandchars=\\\{\}]
\PYG{c}{\PYGZsh{} Files are named like this:}
\PYG{c}{\PYGZsh{} \PYGZlt{}system-date-on-write\PYGZgt{}\PYGZus{}\PYGZlt{}uuid\PYGZgt{}.pickle}
2012-06-15T22:10:29\PYGZus{}7cc2292a-80a6-4fcf-98fc-376953b387ca.pickle
2012-06-15T22:10:41\PYGZus{}e2b1ebb2-1b13-4fb4-bd1c-7fe06aff2758.pickle
2012-06-15T23:04:35\PYGZus{}59dc7790-5f65-47af-99fe-099610099ea4.pickle
2012-06-15T23:04:36\PYGZus{}e58bf4c4-2639-4950-a788-6c84e1c4d1a6.pickle
2012-06-15T23:04:51\PYGZus{}360107b5-d946-4c66-95c8-0d6ceb7a8c8a.pickle
...
\end{Verbatim}

\emph{Advantages:}
\begin{itemize}
\item {} 
Much faster.

\end{itemize}

\emph{Disadvantages:}
\begin{itemize}
\item {} 
Changes in the internal representation may break things

\item {} 
If Base-Attributes of the XML Files are changed manually, they will not be found.

\end{itemize}
\end{quote}


\section{Implementation}
\label{recover:implementation}
\textbf{Actual functions to use:}
\index{rebuild() (in module dbrecover.recover)}

\begin{fulllineitems}
\phantomsection\label{recover:dbrecover.recover.rebuild}\pysiglinewithargsret{\code{dbrecover.recover.}\bfcode{rebuild}}{}{}
Rebuilds the db either by using PickleDBRecover or XMLDBRecover

\end{fulllineitems}

\index{remove() (in module dbrecover.recover)}

\begin{fulllineitems}
\phantomsection\label{recover:dbrecover.recover.remove}\pysiglinewithargsret{\code{dbrecover.recover.}\bfcode{remove}}{}{}
Removes db

\end{fulllineitems}

\index{repair() (in module dbrecover.repair)}

\begin{fulllineitems}
\phantomsection\label{recover:dbrecover.repair.repair}\pysiglinewithargsret{\code{dbrecover.repair.}\bfcode{repair}}{}{}
Walks through domain hierarchy invoking repair() and clear\_locks()

\end{fulllineitems}



\bigskip\hrule{}\bigskip

\index{XMLDBRecover (class in dbrecover.xml\_recover)}

\begin{fulllineitems}
\phantomsection\label{recover:dbrecover.xml_recover.XMLDBRecover}\pysigline{\strong{class }\code{dbrecover.xml\_recover.}\bfcode{XMLDBRecover}}
XMLDBRecover submodule class
\index{description (dbrecover.xml\_recover.XMLDBRecover attribute)}

\begin{fulllineitems}
\phantomsection\label{recover:dbrecover.xml_recover.XMLDBRecover.description}\pysigline{\bfcode{description}}~\begin{quote}\begin{description}
\item[{Returns}] \leavevmode
module description

\end{description}\end{quote}

\end{fulllineitems}

\index{load() (dbrecover.xml\_recover.XMLDBRecover method)}

\begin{fulllineitems}
\phantomsection\label{recover:dbrecover.xml_recover.XMLDBRecover.load}\pysiglinewithargsret{\bfcode{load}}{}{}
Invokes threaded xml recovery

\end{fulllineitems}

\index{recover\_domain() (dbrecover.xml\_recover.XMLDBRecover method)}

\begin{fulllineitems}
\phantomsection\label{recover:dbrecover.xml_recover.XMLDBRecover.recover_domain}\pysiglinewithargsret{\bfcode{recover\_domain}}{\emph{domain}}{}
Iterates through given domain trying to recover metadata

\end{fulllineitems}


\end{fulllineitems}



\bigskip\hrule{}\bigskip

\index{PickleDBRecover (class in dbrecover.pickle\_recover)}

\begin{fulllineitems}
\phantomsection\label{recover:dbrecover.pickle_recover.PickleDBRecover}\pysigline{\strong{class }\code{dbrecover.pickle\_recover.}\bfcode{PickleDBRecover}}
Recovers database from previously generated pickle files
\index{description (dbrecover.pickle\_recover.PickleDBRecover attribute)}

\begin{fulllineitems}
\phantomsection\label{recover:dbrecover.pickle_recover.PickleDBRecover.description}\pysigline{\bfcode{description}}~\begin{quote}\begin{description}
\item[{Returns}] \leavevmode
description

\end{description}\end{quote}

\end{fulllineitems}

\index{load() (dbrecover.pickle\_recover.PickleDBRecover method)}

\begin{fulllineitems}
\phantomsection\label{recover:dbrecover.pickle_recover.PickleDBRecover.load}\pysiglinewithargsret{\bfcode{load}}{}{}
Loads pickle files and regenerates metalist
\begin{quote}\begin{description}
\item[{Returns}] \leavevmode
metalist object

\end{description}\end{quote}

\end{fulllineitems}

\index{save() (dbrecover.pickle\_recover.PickleDBRecover method)}

\begin{fulllineitems}
\phantomsection\label{recover:dbrecover.pickle_recover.PickleDBRecover.save}\pysiglinewithargsret{\bfcode{save}}{\emph{metalist}}{}
Dumps given metalist as pickle file

\end{fulllineitems}


\end{fulllineitems}



\chapter{Indices and tables}
\label{index:indices-and-tables}\begin{itemize}
\item {} 
\emph{genindex}

\item {} 
\emph{modindex}

\item {} 
\emph{search}

\end{itemize}


\renewcommand{\indexname}{Python Module Index}
\begin{theindex}
\def\bigletter#1{{\Large\sffamily#1}\nopagebreak\vspace{1mm}}
\bigletter{c}
\item {\texttt{crawler.dbgen}}, \pageref{dbgen:module-crawler.dbgen}
\item {\texttt{crawler.git}}, \pageref{git:module-crawler.git}
\end{theindex}

\renewcommand{\indexname}{Index}
\printindex
\end{document}
