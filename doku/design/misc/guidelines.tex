\chapter{Werzeuge und Entwicklungsumgebung}
\label{cha:werzeuge_und_entwicklungsumgebung}

\section{Toolauswahl} 
\label{sec:toolauswahl}

\begin{description}
    \item [git:] wird als SCM Tool verwendet, github wird hier als 
        Hostingplattform verwendet \url{github.com/studentkitten/webarchive}
    \item [ant:] Buildsystem für Java-Komponenten
    \item [javadoc:] automatische Generierung der Java Dokumentation
    \item [sphinx:] halbautomatische Generierung der Python Dokumentation
    \item [junit:] Testframework für Java
    \item [unittest:] Python internes Testframework 
    \item [latex:] allgemeine Dokumentation
    
\end{description}


\section{Systemkomponenten} 
\label{sec:systemkomponenten}
\begin{description}
    \item [wget:] Crawler-Kernkomponente
    \item [git:] Versionierung des Archivs
    \item [rsync:] Synchronisation von Crawldaten ins Archiv
\end{description}


Die Gruppe hat für die Entwicklung zwei Untergruppen gebildet, 
eine für das Python Backend (\ciii, \flo, \ci) und eine für den Java Frontend
(\cii, \sab, \eddy).

\section{Programmierrichtlinien} 
\label{sec:guidelines}
\subsection{Python} 
	\url{http://www.python.org/dev/peps/pep-0008/}, \\alternativ \texttt{echo 'import this' | python}
\subsection{Java}
	  	
\footnote{ Originalschema von \href{http://www.home.hs-karlsruhe.de/~pach0003/informatik\_1/java\_richtlinien/einleitung.html}
	{www.home.hs-karlsruhe.de/~pach0003/informatik\_1/java\_richtlinien/einleitung.html} kopiert und angepasst.
}
\begin{description}
\item [Bezeichner]
	Aussagekräftige Namen, ,,sprechende Namen'' z.B. \code{SelectCommitTag} \\
	Ausschließlich in englischer Sprache \\
    Allgemeine Schreibweise in CamelCase \\
\item [Klassen]
	erster Buchstabe groß, z.B. \code{MetaData} \\
\item [Kontrollstrukturen]
	Generell in Blockschreibweise, auch wenn Anweisungen verwendet werden können. \\
	Klammern von Bedingungen mit Leerzeichen absätzen. \\
	Geschweifte Klammern kompakt setzen, bei else oder else if eine Zeile verwenden. \\
	Beispiel:
	\begin{lstlisting}
		if (...) {
			... 
		} else {
			...
		}
	\end{lstlisting}
	Schleifen: for-Schleifen nur verwenden wenn die Anzahl der Durchläufe vorher bekannt ist, sonst (do-)while.
\item [Variablen, Attribute und Parameter]
	erster Buchstabe klein, bsp. \code{countClients} \\
    i,j,k,l für Zählvariablen verwenden
	
\item [Konstanten und enums]
	Konstanten werden komplett groß geschrieben, Trennung erfolgt mit dem Unterstrich, bsp. \code{MESSAGE\_HEADER}. \\
	Bei mehreren symbolischen Werten wie z.B. Zustände sind enums zu bevorzugen. \\
	Beispiel:
	\code{enum State \{RUNNING, PAUSE, STOPPED\}}

\item [Ausdrücke]
	Binäre Operatoren mit führendem und nachfolgendem Leerzeichen schreiben. \\

\item [Methoden]
	mindestens ein Verb im Namen der Methode verwenden: bsp. \code{select()}, \code{selectAll()} \\
	Getter und setter setzen sich aus Verb und Membername zusammen: bsb. \\
	\begin{lstlisting}
		class Person {
			private String name;
			public String getName() {...}
			public void setName() {...}
		}
	\end{lstlisting}
	Kurze Methoden mit nicht mehr als 25 Zeilen schreiben. 

\item [Static]
	Klassenvariablen und -methoden werden immer mit Klassennamen referenziert. \\ 
	Bsp \code{Math.abs(-16)}, \code{Calendar.YEAR}
\item [Packages]
    Nur Kleinbuchstaben für Paketnamen verwenden. \\
    Nie java oder javax für Paketnamen. \\
	Pakete sinnvoll nach Komponenten und Funktionen gliedern.
\item [Quelltextformatierung]
\begin{description}
	\item [Allgemein]
		Nicht mehr als 80 Zeichen pro Zeile
	\item [Einrückungen]
		Tabulator verwenden. \\
		Keine Ersetzung durch Leerzeichen.
\end{description}
\item [Vor- , Nachbedingungen und Invarianten]
	Vor- , Nachbedingungen und Invarianten also unveränderliche Bedingungen vor dem Beginn,
	innerhalb und vor dem Verlassen von Methoden sind mit assert-Statements zu prüfen. \\
	Bsp. Vorbedingung
	\begin{lstlisting}
		private div(long otherValue) {
			assert otherValue != 0;
			value /= otherValue;
		}
	\end{lstlisting}
	Triviale Bedingungen müssen nicht geprüft werden. \\
	Beim Ausführen auf der VM sind während der Entwicklungsphase asserts einzuschalten.

\item [Javadoc und Kommentare]

\begin{description}
	\item [Allgemeines]
	    Beschreiben was eine Methode macht, nicht wie sie es macht\\
	    Kurz aber spezifisch
	\item [Klassen]
		Allgemeine Beschreibung der Klassenfunktion.\\
	    Entwurfsentscheidungen im Klassenkommentar dokumentieren.
	\item [Attribute]
    	Bedeutung von Attributen und Variablen dokumentieren 
	\item [Methoden]
		Erster Satz: Allgemeine Beschreibung \\
	\item [Return Wert]
		Beschreibung was Zurückgegeben wird (evtl. mit Fallunterscheidung)
	\item [Exceptions]
		Grund angeben, falls bekannt.
	\item [private Methoden]
		Nur erforderlich falls die Methode komplex ist (z.B. viele Parameter)
		und Zweck nicht aus den Namen hervorgeht.
	\item [Kommentare im Quelltext]
		Nur erforderlich bei komplexen Codeabschnitten deren Sinn nicht
		unmittelbar erkennbar ist, z.B. bei optiemierten Code.
\end{description}
\end{description}
  	  	 



%\subsection{Tests}
%    Innerhalb der Untergruppen werden die Module gegenseitig getestet um das Risiko von ,,Denkfehlern'' zu minimieren


% section systemkomponenten (end)
