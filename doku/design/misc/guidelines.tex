\chapter{Werzeuge und Entwicklungsumgebung}
\label{cha:werzeuge_und_entwicklungsumgebung}

\section{Toolauswahl} 
\label{sec:toolauswahl}

\begin{description}
    \item [git:] wird als SCM Tool verwendet, github wird hier als 
        Hostingplattform verwendet \url{github.com/studentkitten/webarchive}
    \item [ant:] Buildsystem für Java-Komponenten
    \item [javadoc:] automatische Generierung der Java Dokumentation
    \item [sphinx:] halbautomatische Generierung der Python Dokumentation
    \item [junit:] Testframework für Java
    \item [unittest:] Python internes Testframework 
    \item [latex:] allgemeine Dokumentation
    
\end{description}


\section{Systemkomponenten} 
\label{sec:systemkomponenten}
\begin{description}
    \item [wget:] Crawler-Kernkomponente
    \item [git:] Versionierung des Archivs
    \item [rsync:] Synchronisation von Crawldaten ins Archiv
\end{description}


Die Gruppe hat für die Entwicklung zwei Untergruppen gebildet, 
eine für das Python Backend (\ciii, \flo, \ci) und eine für den Java Frontend
(\cii, \sab, \eddy).

\section{Programmierrichtlinien} 
\label{sec:guidelines}
\begin{description}
    \item [Python:] \url{http://www.python.org/dev/peps/pep-0008/}, \\alternativ \texttt{echo 'import this' | python}
    \item [Java:] Spaghetticode Guideline TODO
    \item [Allgemein:] In den Untergruppen werden die Module gegenseitig getestet um das Risiko von ,,Denkfehlern'' zu minimieren
\end{description}


% section systemkomponenten (end)
